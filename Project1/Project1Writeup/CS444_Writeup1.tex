\documentclass[letterpaper,10pt,titlepage]{article}

\setlength{\parindent}{0pt}

\usepackage{graphicx}
\usepackage{amssymb}
\usepackage{amsmath}
\usepackage{amsthm}

\usepackage{alltt}
\usepackage{float}
\usepackage{color}
\usepackage{url}
\usepackage{listings}

\usepackage{balance}
\usepackage[TABBOTCAP, tight]{subfigure}
\usepackage{enumitem}
\usepackage{pstricks, pst-node}

\usepackage{geometry}
\geometry{textheight=8.5in, textwidth=6in}

\newcommand{\cred}[1]{{\color{red}#1}}
\newcommand{\cblue}[1]{{\color{blue}#1}}

\usepackage{hyperref}
\usepackage{geometry}
\usepackage{vhistory}

\hypersetup{%
	colorlinks = true,
	linkcolor = black
}

\lstdefinestyle{customc}{
  belowcaptionskip=1\baselineskip,
  breaklines=true,
  frame=L,
  xleftmargin=\parindent,
  language=C,
  showstringspaces=false,
  basicstyle=\footnotesize\ttfamily,
  keywordstyle=\bfseries\color{green!40!black},
  commentstyle=\itshape\color{purple!40!black},
  identifierstyle=\color{blue},
  stringstyle=\color{orange},
}

\def\name{Austin Nguyen, Isaac Stallcup}

%pull in the necessary preamble matter for pygments output
\usepackage{fancyvrb}
\usepackage{color}
\usepackage[latin1]{inputenc}


\makeatletter
\def\PY@reset{\let\PY@it=\relax \let\PY@bf=\relax%
    \let\PY@ul=\relax \let\PY@tc=\relax%
    \let\PY@bc=\relax \let\PY@ff=\relax}
\def\PY@tok#1{\csname PY@tok@#1\endcsname}
\def\PY@toks#1+{\ifx\relax#1\empty\else%
    \PY@tok{#1}\expandafter\PY@toks\fi}
\def\PY@do#1{\PY@bc{\PY@tc{\PY@ul{%
    \PY@it{\PY@bf{\PY@ff{#1}}}}}}}
\def\PY#1#2{\PY@reset\PY@toks#1+\relax+\PY@do{#2}}

\expandafter\def\csname PY@tok@gd\endcsname{\def\PY@tc##1{\textcolor[rgb]{0.63,0.00,0.00}{##1}}}
\expandafter\def\csname PY@tok@gu\endcsname{\let\PY@bf=\textbf\def\PY@tc##1{\textcolor[rgb]{0.50,0.00,0.50}{##1}}}
\expandafter\def\csname PY@tok@gt\endcsname{\def\PY@tc##1{\textcolor[rgb]{0.00,0.25,0.82}{##1}}}
\expandafter\def\csname PY@tok@gs\endcsname{\let\PY@bf=\textbf}
\expandafter\def\csname PY@tok@gr\endcsname{\def\PY@tc##1{\textcolor[rgb]{1.00,0.00,0.00}{##1}}}
\expandafter\def\csname PY@tok@cm\endcsname{\let\PY@it=\textit\def\PY@tc##1{\textcolor[rgb]{0.25,0.50,0.50}{##1}}}
\expandafter\def\csname PY@tok@vg\endcsname{\def\PY@tc##1{\textcolor[rgb]{0.10,0.09,0.49}{##1}}}
\expandafter\def\csname PY@tok@m\endcsname{\def\PY@tc##1{\textcolor[rgb]{0.40,0.40,0.40}{##1}}}
\expandafter\def\csname PY@tok@mh\endcsname{\def\PY@tc##1{\textcolor[rgb]{0.40,0.40,0.40}{##1}}}
\expandafter\def\csname PY@tok@go\endcsname{\def\PY@tc##1{\textcolor[rgb]{0.50,0.50,0.50}{##1}}}
\expandafter\def\csname PY@tok@ge\endcsname{\let\PY@it=\textit}
\expandafter\def\csname PY@tok@vc\endcsname{\def\PY@tc##1{\textcolor[rgb]{0.10,0.09,0.49}{##1}}}
\expandafter\def\csname PY@tok@il\endcsname{\def\PY@tc##1{\textcolor[rgb]{0.40,0.40,0.40}{##1}}}
\expandafter\def\csname PY@tok@cs\endcsname{\let\PY@it=\textit\def\PY@tc##1{\textcolor[rgb]{0.25,0.50,0.50}{##1}}}
\expandafter\def\csname PY@tok@cp\endcsname{\def\PY@tc##1{\textcolor[rgb]{0.74,0.48,0.00}{##1}}}
\expandafter\def\csname PY@tok@gi\endcsname{\def\PY@tc##1{\textcolor[rgb]{0.00,0.63,0.00}{##1}}}
\expandafter\def\csname PY@tok@gh\endcsname{\let\PY@bf=\textbf\def\PY@tc##1{\textcolor[rgb]{0.00,0.00,0.50}{##1}}}
\expandafter\def\csname PY@tok@ni\endcsname{\let\PY@bf=\textbf\def\PY@tc##1{\textcolor[rgb]{0.60,0.60,0.60}{##1}}}
\expandafter\def\csname PY@tok@nl\endcsname{\def\PY@tc##1{\textcolor[rgb]{0.63,0.63,0.00}{##1}}}
\expandafter\def\csname PY@tok@nn\endcsname{\let\PY@bf=\textbf\def\PY@tc##1{\textcolor[rgb]{0.00,0.00,1.00}{##1}}}
\expandafter\def\csname PY@tok@no\endcsname{\def\PY@tc##1{\textcolor[rgb]{0.53,0.00,0.00}{##1}}}
\expandafter\def\csname PY@tok@na\endcsname{\def\PY@tc##1{\textcolor[rgb]{0.49,0.56,0.16}{##1}}}
\expandafter\def\csname PY@tok@nb\endcsname{\def\PY@tc##1{\textcolor[rgb]{0.00,0.50,0.00}{##1}}}
\expandafter\def\csname PY@tok@nc\endcsname{\let\PY@bf=\textbf\def\PY@tc##1{\textcolor[rgb]{0.00,0.00,1.00}{##1}}}
\expandafter\def\csname PY@tok@nd\endcsname{\def\PY@tc##1{\textcolor[rgb]{0.67,0.13,1.00}{##1}}}
\expandafter\def\csname PY@tok@ne\endcsname{\let\PY@bf=\textbf\def\PY@tc##1{\textcolor[rgb]{0.82,0.25,0.23}{##1}}}
\expandafter\def\csname PY@tok@nf\endcsname{\def\PY@tc##1{\textcolor[rgb]{0.00,0.00,1.00}{##1}}}
\expandafter\def\csname PY@tok@si\endcsname{\let\PY@bf=\textbf\def\PY@tc##1{\textcolor[rgb]{0.73,0.40,0.53}{##1}}}
\expandafter\def\csname PY@tok@s2\endcsname{\def\PY@tc##1{\textcolor[rgb]{0.73,0.13,0.13}{##1}}}
\expandafter\def\csname PY@tok@vi\endcsname{\def\PY@tc##1{\textcolor[rgb]{0.10,0.09,0.49}{##1}}}
\expandafter\def\csname PY@tok@nt\endcsname{\let\PY@bf=\textbf\def\PY@tc##1{\textcolor[rgb]{0.00,0.50,0.00}{##1}}}
\expandafter\def\csname PY@tok@nv\endcsname{\def\PY@tc##1{\textcolor[rgb]{0.10,0.09,0.49}{##1}}}
\expandafter\def\csname PY@tok@s1\endcsname{\def\PY@tc##1{\textcolor[rgb]{0.73,0.13,0.13}{##1}}}
\expandafter\def\csname PY@tok@sh\endcsname{\def\PY@tc##1{\textcolor[rgb]{0.73,0.13,0.13}{##1}}}
\expandafter\def\csname PY@tok@sc\endcsname{\def\PY@tc##1{\textcolor[rgb]{0.73,0.13,0.13}{##1}}}
\expandafter\def\csname PY@tok@sx\endcsname{\def\PY@tc##1{\textcolor[rgb]{0.00,0.50,0.00}{##1}}}
\expandafter\def\csname PY@tok@bp\endcsname{\def\PY@tc##1{\textcolor[rgb]{0.00,0.50,0.00}{##1}}}
\expandafter\def\csname PY@tok@c1\endcsname{\let\PY@it=\textit\def\PY@tc##1{\textcolor[rgb]{0.25,0.50,0.50}{##1}}}
\expandafter\def\csname PY@tok@kc\endcsname{\let\PY@bf=\textbf\def\PY@tc##1{\textcolor[rgb]{0.00,0.50,0.00}{##1}}}
\expandafter\def\csname PY@tok@c\endcsname{\let\PY@it=\textit\def\PY@tc##1{\textcolor[rgb]{0.25,0.50,0.50}{##1}}}
\expandafter\def\csname PY@tok@mf\endcsname{\def\PY@tc##1{\textcolor[rgb]{0.40,0.40,0.40}{##1}}}
\expandafter\def\csname PY@tok@err\endcsname{\def\PY@bc##1{\setlength{\fboxsep}{0pt}\fcolorbox[rgb]{1.00,0.00,0.00}{1,1,1}{\strut ##1}}}
\expandafter\def\csname PY@tok@kd\endcsname{\let\PY@bf=\textbf\def\PY@tc##1{\textcolor[rgb]{0.00,0.50,0.00}{##1}}}
\expandafter\def\csname PY@tok@ss\endcsname{\def\PY@tc##1{\textcolor[rgb]{0.10,0.09,0.49}{##1}}}
\expandafter\def\csname PY@tok@sr\endcsname{\def\PY@tc##1{\textcolor[rgb]{0.73,0.40,0.53}{##1}}}
\expandafter\def\csname PY@tok@mo\endcsname{\def\PY@tc##1{\textcolor[rgb]{0.40,0.40,0.40}{##1}}}
\expandafter\def\csname PY@tok@kn\endcsname{\let\PY@bf=\textbf\def\PY@tc##1{\textcolor[rgb]{0.00,0.50,0.00}{##1}}}
\expandafter\def\csname PY@tok@mi\endcsname{\def\PY@tc##1{\textcolor[rgb]{0.40,0.40,0.40}{##1}}}
\expandafter\def\csname PY@tok@gp\endcsname{\let\PY@bf=\textbf\def\PY@tc##1{\textcolor[rgb]{0.00,0.00,0.50}{##1}}}
\expandafter\def\csname PY@tok@o\endcsname{\def\PY@tc##1{\textcolor[rgb]{0.40,0.40,0.40}{##1}}}
\expandafter\def\csname PY@tok@kr\endcsname{\let\PY@bf=\textbf\def\PY@tc##1{\textcolor[rgb]{0.00,0.50,0.00}{##1}}}
\expandafter\def\csname PY@tok@s\endcsname{\def\PY@tc##1{\textcolor[rgb]{0.73,0.13,0.13}{##1}}}
\expandafter\def\csname PY@tok@kp\endcsname{\def\PY@tc##1{\textcolor[rgb]{0.00,0.50,0.00}{##1}}}
\expandafter\def\csname PY@tok@w\endcsname{\def\PY@tc##1{\textcolor[rgb]{0.73,0.73,0.73}{##1}}}
\expandafter\def\csname PY@tok@kt\endcsname{\def\PY@tc##1{\textcolor[rgb]{0.69,0.00,0.25}{##1}}}
\expandafter\def\csname PY@tok@ow\endcsname{\let\PY@bf=\textbf\def\PY@tc##1{\textcolor[rgb]{0.67,0.13,1.00}{##1}}}
\expandafter\def\csname PY@tok@sb\endcsname{\def\PY@tc##1{\textcolor[rgb]{0.73,0.13,0.13}{##1}}}
\expandafter\def\csname PY@tok@k\endcsname{\let\PY@bf=\textbf\def\PY@tc##1{\textcolor[rgb]{0.00,0.50,0.00}{##1}}}
\expandafter\def\csname PY@tok@se\endcsname{\let\PY@bf=\textbf\def\PY@tc##1{\textcolor[rgb]{0.73,0.40,0.13}{##1}}}
\expandafter\def\csname PY@tok@sd\endcsname{\let\PY@it=\textit\def\PY@tc##1{\textcolor[rgb]{0.73,0.13,0.13}{##1}}}

\def\PYZbs{\char`\\}
\def\PYZus{\char`\_}
\def\PYZob{\char`\{}
\def\PYZcb{\char`\}}
\def\PYZca{\char`\^}
\def\PYZam{\char`\&}
\def\PYZlt{\char`\<}
\def\PYZgt{\char`\>}
\def\PYZsh{\char`\#}
\def\PYZpc{\char`\%}
\def\PYZdl{\char`\$}
\def\PYZti{\char`\~}
% for compatibility with earlier versions
\def\PYZat{@}
\def\PYZlb{[}
\def\PYZrb{]}
\makeatother

%% The following metadata will show up in the PDF properties
\hypersetup{
  colorlinks = true,
  urlcolor = black,
  pdfauthor = {\name},
  pdfkeywords = {CS444 ``Operating Systems''},
  pdftitle = {CS 444 Writeup 1},
  pdfsubject = {CS 444 Writeup 1},
  pdfpagemode = UseNone
}

\begin{document}

\begin{titlepage}
    \begin{center}
        \vspace*{3.5cm}

        \textbf{Writeup 1}

        \vspace{0.5cm}
	
	\vspace{0.8cm}

        CS 444\\
        Spring 2017\\

        \vspace{1cm}

        \textbf{Abstract}\\

        \vspace{0.5cm}

	This writeup is the writeup for Project 1 for CS444 Operating Systems 2.  The write-up includes the Linux Yocto kernal on os-class.engr.oregonstate.edu and the solution of Producer-Consumer in C.

        \vfill

    \end{center}
\end{titlepage}

\newpage

\tableofcontents

\newpage

\section{Concurrency writeup}
The producer-consumer problem is a well studied problem.  Algorithms were simply found from a search and used the POSIX threads and lpthread build from the linux command line.  After producing and creating thread functions were created, these were used to pass items back and forth before joining.  In order to prevent simultaneous access, mutexes were used, along with random waiting period, where threads would be blocked from accessing the array of items if another thread was accessing at the time.

\section{Log of Commands to Build Yocto Kernel and Load Qemu}
\begin{lstlisting}
1.	Open two terminal windows. In terminal #1, do steps 2-12. In terminal #2 do steps 2-3. In terminal #1 we will build our project and use gdb to control the emulator in the debug mode. In terminal #2, we will boot our kernels on the VM.
2.	Log on to os-class : pengc@os-class.oregonstate.edu(Please use your own user name )
3.	Call cd scratch/spring2017/10-01 . If you don't have this folder(10-01), call mkdir 10-01 to create under /scratch/spring2017/ directory and cd 10-01 to change the directory
4.	Call git clone git://git.yoctoproject.org/linux-yocto-3.14 to download the project from GitHub and you will get linux-yocto-3.14.
5.	To switch to the correct tag, call cd linux-yocto-3.14, and then git checkout v3.14.26 
6.	Before we build our kernel or run qemu, we should configure the environment, so run source /scratch/opt/environment-setup-i586-poky-linux.csh 
7.	Follow 8-12 steps, make a kernel instance for your group.
8.	Run cp /scratch/spring2017/files/config-3.14.26-yocto-qemu .config
9.	Run make menuconfig and you will get a window
10.	In the widow do the following: press / and type in LOCALVERSION, press enter. 
11.	Hit 1, press enter and then edit the value to be -10-01-hw1 (-10-01-hw1 for group 10-01). This will be appended to the kernel name
12.	Run make -j4 all, your kernel instance will be built with 4 threads
13.	Run cd .. and then run gdb. Stop here for now.
14.	In terminal #2, do step 6.
15.	To make a copy for the starting kernel and the drive file located in /scratch/spring2017/files/, do steps 16-17 under your group directory, ex. /scratch/spring2017/10-01 
16.	Call cp /scratch/spring2017/files/bzImage-qemux86.bin . (This . is an operand)
17.	Call  /scratch/spring2017/files/core-image-lsb-sdk-qemux86.ext3 . (This . is an operand)
18.	Try run the starting kernel : Call qemu-system-i386 -gdb tcp::5601 -S -nographic -kernel bzImage-qemux86.bin -drive file=core-image-lsb-sdk-qemux86.ext3,if=virtio -enable-kvm -net none -usb -localtime --no-reboot --append "root=/dev/vda rw console=ttyS0 debug" (Here I use 5601 because I took an example of group 10-01, the port number should always be 5600+ some #. In this case, it is 5600 +101.)
19.	Since in step 18, we run qemu in debug mode with the CPU halted. We need to use gdb to control it. Do steps 20-21.
20.	In terminal #1, it is now in gdb. Run target remote :5601 to connect the qemu.
21.	Run continue. Then you will see the change in terminal #2. 
22.	If you succeed in running qemu, you will be asked to login. Type root and enter. Run uname -a and you will see that the kernel name.
23.	Use reboot to reboot the VM.
24.	Try run the kernel instance we created in steps 8-12. The kernel instance we built is located in linux-yocto-3.14/arch/x86/boot/ and it is named bzImage.
25.	Run qemu-system-i386 -gdb tcp::5601 -S -nographic -kernel linux-yocto-3.14/arch/x86/boot/bzImage  -drive file=core-image-lsb-sdk-qemux86.ext3,if=virtio -enable-kvm -net none -usb -localtime --no-reboot --append "root=/dev/vda rw console=ttyS0 debug"
26.	Do steps 20-22 again. You should find the difference in kernel names. The name should be customized into something like ??-10-01-hw1, because uname -a produce the LOCALVERSION string.
27.	reboot the vm and use q to quit gdb.
\end{lstlisting}

\section{Flags in the listed Qemu command line}
The listed Qemu command line is:
\begin{lstlisting}
qemu-system-i386 -gdb tcp::???? -S -nographic -kernel bzImage-qemux86.bin 
\end{lstlisting}
\begin{lstlisting}
-drive file=core-image-lsb-sdk-qemux86.ext3,if=virtio -enable-kvm 
\end{lstlisting}
\begin{lstlisting}
-net none -usb -localtime --no-reboot --append 
\end{lstlisting}
\begin{lstlisting}
"root=/dev/vda rw console=ttyS0 debug".
\end{lstlisting}

The following list describes each flag:
\begin{itemize}
\item \emph{-gdb} instructs qemu to wait for input from the subsequent flag.
\item \emph{tcp::5628} gives qemu the device to listen on for gdb input.
\item \emph{-S} tells qemu to not start the CPU at startup.
\item \emph{-nographic} disables graphical output so qemu appears as a command line application only.
\item \emph{-kernel} followed by a file specifies the kernel to be used, in this case the bzImage we created.
\item \emph{-drive} defines a new drive; the following arguemnts specify the disk image file (file=core-image-lsb-sdk-qemux86.ext3) and the interface option (if=virtio).
\item \emph{-enable-kvm} enables KVM virtualization, a very complicated function that assists in emulating hardware.
\item \emph{-net} interfaces with network options; assumedly as it is followed by "none" the VM does not have networking support. 
\item \emph{-usb} enables the USB driver.
\item \emph{-localtime} enables the machine to start at the current local time.
\item \emph{--no-reboot} configures the system to shutoff instead of reboot.
\item \emph{--append} tells qemu to use the following argument ("root=dev/vdarw console=tty50 debug") as the kernel command line.
\end{itemize}

\section{Concurrency}
The following subsections answer the four questions as outlined on the Project 1 page on Kevin McGrath's course website.
\subsection{Main point of assignment}
The main points of the assignment were to remind us and refresh our usage of the pthreads in C which we did one (maybe 2?) assignments in CS344 which was probably a year removed from most of us in the class. In addition, it showed us a practical way the multiparallel processing is useful in a possible real life scenario of producer-consumer, of course in which real life is much more complicated.  Third, if some of us had not programmed in C in a while (my last couple CS classes have been in python and java), it was a good refresher for that as well.

\subsection{Personal approach to problem}
My approach comes from some research with the producer-consumer problem.  The pthreads documentation is pretty easy to find so functions such as create, join, mutex, etc. were all used in thsi problem.  Because the easiest way to make the buffer was a array of structs, it made more sense to use static threads in which the buffer has 2 data fields, a random val and a random waiting time.  

My approach for the use of threads came from guidelines to the canonical Producer-Consumer problem that I found in the \emph{Linux Programming Interface} text. For example, statements like pthread\_create, pthread\_join, pthread\_t, pthread\_mutex\_t, and pthread\_cond\_t were all found in that book and that book instructed me on how to use them. I used these static pthread statements rather than dynamic ones because frankly, I could not think of a reason why dynamic ones would be necessary for the purposes of this assignment. My approach for the buffer and the item to be held in the buffer was just to use a struct to represent the buffer and a struct to represent the item and have the buffer hold a statically allocated array of pointers to items and for each item to have two data fields: a random number and a random waiting period, both that had to be implemented in ASM. Given the statements necessary to control threads and mutexes and the structs necessary to store data, I solved the problem by creating a thread to correspond to a Producer, having that Producer thread lock all of the shared data using a mutex and then produce a random item. After unlocking the shared data within the Producer thread, I created a Consumer thread that locked the data again, consumed the first non-null index in the buffer, and then unlocked the data again. I contained these Producer and Consumer thread creation statements in a for loop that runs the number of times the user indicates in a command line argument so that when the program is run it creates the corresponding producer and consumer threads and produces and consumes random data. 
\subsection{Ensuring solution was correct}
To ensure that the buffer was properly being added/removed to and to ensure that both threads were properly locking and unlocking, I added print statements in important functions such as when the global mutex was locked and unlocked in order to know whether these calls were actually being made. It is easy to verify that the threads are switching back and forth because both threads have different print statements that are only accessed within their specific thread is running. It is also easy to realize that the buffer is only sequentially being accessed by the two different threads in accordance with the mutex locking and unlocking because otherwise both threads would be accessing the same resource and either a memory read/write issue would occur or both threads would be depending on each other for an inordinate amount of time, resulting in \emph{deadlock}. Obviously, my use of the pthread condition functionality allowed both threads to wait until resources were available, thus preventing deadlock. Finally, to ensure that the entire integrated software worked properly, I used a makefile to generate an executable of the code that accepts 1 argument that corresponds to the number of times a loop that creates the threads shall occur. To ensure the executable achieved what I wanted it to, I ran the software with command line arguments of 1, 2, 3, 4, and 5. The corresponding output for these arguments properly demonstrated that the different threads were called 1, 2, 3, 4, and 5 times, respectively, which is correct behavior.

\subsection{What I learned}
Austin reviewed knowledge of concurrency and with the synchro2 problem discussed in the recitation, is starting to get into the parallel mindset. Using mutexes again helped this process to remember to not allow multiple threads to access the same data.  The assignment was a good warmup for any future problems in parallel.


\section{Version control log}
\begin{versionhistory}
\subsection{Git Logs}
8badf40 was Austin, 8 minutes ago, message: pdf test creation
597d814 was Austin, 43 minutes ago, message: Tex edit

304e7e8 was IStallcup, 4 hours ago, message: small changes to concurrency

5336067 was IStallcup, 5 hours ago, message: Merge branch 'master' of https://github.com/Boeinco/CS444
6
fa3edf was IStallcup, 5 hours ago, message: added qemu flag explanation

5ffd876 was Isaac Stallcup, 5 hours ago, message: Create .gitignore

66d601f was Austin, 12 hours ago, message: Writeup work

53de57c was Austin, 12 hours ago, message: fixed command arg
c
fb1942 was Austin, 32 hours ago, message: writeup updated

401f05e was Austin, 32 hours ago, message: writeup updated

8425628 was IStallcup, 33 hours ago, message: Added some more stuff

cf27ae5 was IStallcup, 33 hours ago, message: added some writeup formatting

6e073c6 was Austin, 34 hours ago, message: Template tex file done

dae7e2e was Austin, 35 hours ago, message: concurrency mostly done

4d7f1cc was Austin, 2 days ago, message: temp files

3e7be70 was Austin, 2 days ago, message: some files in

599f66e was Austin, 2 days ago, message: started concurrency

c6a4f0c was Austin, 2 days ago, message: Testing directories

8e404a9 was Austin Nguyen, 3 days ago, message: first commit

\end{versionhistory}

\section{Work log}
\subsection{April 18-17}
Austin created git project and started working on concurrency.  Used latex template to set up the latex document and git commits under username Boeinco.
\subsection{April 19-17}
Austin started working on the new project requirement of the command line argument after creating the main project frame.  After creating mutexes and having them sleep for the specified random amount of time, work was done on translating it to the producer/consumer problem.
\subsection{April 20-17}
Austin finished up the producer/consumer problem and added the qemu instructions. Isaac did research on the flags during the linux commands and also helped format the latex file.  Some latex work was done.
\subsection{April 21-17}
Austin did the section of version control and other parts of the document.

%\bibliographystyle{plain}
%\bibliography{CS444_Writeup1}
\end{document}
